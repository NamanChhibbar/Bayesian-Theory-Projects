\section{Metrics}

\begin{frame}{Reporting Posterior Metrics}

  \begin{itemize}
    \item By the end of 10 years, we report key metrics from the final posterior distribution.
    \item 95\% Confidence Interval: The range where the team’s win probability is most likely to lie.
    \item Posterior Mean: The expected win probability based on all 10 years of data.
    \item Variance: Reflects our confidence in this probability estimate; lower variance implies greater certainty.
  \end{itemize}
  
\end{frame}

\begin{frame}{Posterior Mean and Variance Analysis}

  \begin{itemize}
    \item The posterior mean over time shows how our estimate of win probability has evolved.
    \item A decreasing variance suggests that our confidence is increasing, as we have more data to base our estimates on.
    \item These metrics help us assess how consistent the team’s performance has been over time.
  \end{itemize}
  
\end{frame}

\begin{frame}{Year-by-Year Posterior Distributions}

  \begin{itemize}
    \item Here, we visualize the evolution of the posterior distributions over each year.
    \item These distributions become more concentrated, indicating increased confidence as more data is incorporated over time.
  \end{itemize}
  
\end{frame}
