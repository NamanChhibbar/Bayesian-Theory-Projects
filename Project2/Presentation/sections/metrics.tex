\section{Metrics}

\begin{frame}{Reporting Posterior Metrics}

\begin{itemize}
  \item By the end of the year, we report key metrics from the final posterior distribution.
  \item 95\% Confidence Interval: The range where the product's revenue is most likely to lie.
  \item Posterior Mean: The expected revenue generated based on whole year of data (which is also equal to the median and mode).
  \item Variance: Reflects our confidence in this revenue estimate; lower variance implies greater certainty.
\end{itemize}

\end{frame}

\begin{frame}{95\% Confidence Interval}

\begin{itemize}
  \item Posterior mean after the updates: 33,800
  \item 95\% Confidence Interval: [28974.46, 38,625.54]
\end{itemize}

\begin{figure}
  \centering
  \includegraphics[width=.8\linewidth]{../Report/images/ci.png}
  \caption{95\% Confidence Interval}
\end{figure}
  
\end{frame}

\begin{frame}{Posterior Mean and Variance Analysis}

\begin{itemize}
  \item The posterior mean over time shows how our estimate of revenue has evolved.
  \item A decreasing variance suggests that our confidence is increasing, as we have more data to base our estimates on.
  \item These metrics help us assess how consistent the product sale has been over time.
\end{itemize}

\end{frame}

\begin{frame}{Posterior Mean}

\begin{figure}
  \centering
  \includegraphics[width=.8\linewidth]{../Report/images/mean.png}
  \caption{Posterior Mean for each month}
\end{figure}

\end{frame}

\begin{frame}{Posterior Variance}

\begin{figure}
  \centering
  \includegraphics[width=.8\linewidth]{../Report/images/var.png}
  \caption{Posterior Variance for each month}
\end{figure}

\end{frame}